\chapter{Journal de travail}

\begin{landscape}

\begin{longtable}[c]{lp{10cm}rrrr}
    \caption{Journal de travail}\\

    \hline
    Date & Description & Rech. [h] & Dev. [h] & Rapport [h] & Admin [h] \\
    \hline
    \endfirsthead
    
    \hline
    Date & Description & Rech. [h] & Dev. [h] & Rapport [h] & Admin [h] \\
    \hline
    \endhead
    
    \multicolumn{6}{r}{\small \it Le journal de travail continue à la page suivante.} \\
    \normalsize
    \endfoot
    
    \hline
    \endlastfoot


  % Work
	25.02.2020 
	& Kick-off du projet et rdv avec M. Kapfer
	& 0 %recherche
	& 0 %dev
	& 0 %reporting
	& 2\\ %admin

	04.03.2020 
	& Travail sur le planning et mise en place du journal
	& 0 %recherche
	& 0 %dev
	& 0 %reporting
	& 4\\ %admin

	05.03.2020 
	& Recherche sur les différentes distributions et travail sur le planning
	& 2 %recherche
	& 0 %dev
	& 0 %reporting
	& 2\\ %admin

	06.03.2020 
	& Recherche sur les différentes distributions 
	& 3 %recherche
	& 0 %dev
	& 2 %reporting
	& 0\\ %admin

	10.03.2020 
	& Recherche sur les différentes distributions et début de la comparaison entre virtualisation et émulation
	& 3 %recherche
	& 0 %dev
	& 2 %reporting
	& 1\\ %admin
	
	
	11.03.2020 
	& Comparaison entre virtualisation et émulation
	& 5 %recherche
	& 0 %dev
	& 1 %reporting
	& 0\\ %admin
	
	12.03.2020 
	& Comparaison entre virtualisation et émulation
	& 2 %recherche
	& 0 %dev
	& 3 %reporting
	& 0\\ %admin
	
	17.03.2020 
	& Recherche sur les interactions dans un filesystem
	& 4 %recherche
	& 0 %dev
	& 0 %reporting
	& 0\\ %admin
	
	18.03.2020 
	& Recherche sur les interactions dans un filesystem
	& 5 %recherche
	& 0 %dev
	& 1 %reporting
	& 0\\ %admin
	
	19.03.2020 
	& Recherche sur les interactions dans un filesystem
	& 4 %recherche
	& 0 %dev
	& 2 %reporting
	& 0\\ %admin
	
	01.04.2020 
	& Création de dépôt et sandbox des applications avec GUI
	& 6 %recherche
	& 0 %dev
	& 2 %reporting
	& 0\\ %admin
	
	
	08.04.2020 
	& Recherche sur les interactions dans un filesystem
	& 4 %recherche
	& 0 %dev
	& 2 %reporting
	& 0\\ %admin
	
	09.04.2020 
	& Rédaction du cahier des charges
	& 0 %recherche
	& 0 %dev
	& 5 %reporting
	& 1\\ %admin
	
	10.04.2020 
	& Test des technologies de sandbbox avec Docker et rédaction du cahier des charges
	& 2 %recherche
	& 4 %dev
	& 3 %reporting
	& 1\\ %admin
	
	15.04.2020 
	& Test des technologies de sandbbox avec Docker
	& 2 %recherche
	& 4 %dev
	& 0 %reporting
	& 1\\ %admin
	
	16.04.2020
	& Fin de recherches sur les technologies Docker et QEMU et distribution.
	& 6 %recherche
	& 0 %dev
	& 1 %reporting
	& 1\\ %admin
	
	17.04.2020 
	& Setup du PC prêté par l'école
	& 0 %recherche
	& 6 %dev
	& 0 %reporting
	& 1\\ %admin
	
	22.04.2020
	& Setup du PC mais sans succès et finalisation du cahier des charges
	& 0 %recherche
	& 4 %dev
	& 3 %reporting
	& 1\\ %admin
	
	23.04.2020
	& Retour sur le cahier des charges et sur le planning.
	& 0 %recherche
	& 0 %dev
	& 4 %reporting
	& 2\\ %admin
	
	29.04.2020
	& Retour sur le planning et définitions de parties distinctes. Révision du planning par rapport aux heures de travail. Proof of concept des containers Docker avec GUI. Découverte de snap.
	& 3 %recherche
	& 4 %dev
	& 2 %reporting
	& 1\\ %admin
	
	30.04.2020
	& Recherche sur les interactions avec le filesystem.
	& 5 %recherche
	& 0 %dev
	& 2 %reporting
	& 1\\ %admin
	
	06.05.2020
	& Comparaison snap et Docker. Travail sur un nouveau PC car l'ancien de marchait pas -> il a fallu tout reconfigurer.
	& 5 %recherche
	& 0 %dev
	& 2 %reporting
	& 1\\ %admin
	
	07.05.2020
	& Retour sur le planning et modification complète des tâches à faire. Il a fallu remodeler avec l'approche \textbf{Recherches -> Spécifications -> Implémentations -> Validations}. Conclusion sur Snap vs Docker
	& 1 %recherche
	& 0 %dev
	& 4 %reporting
	& 2\\ %admin
	
	13.05.2020
	& Définitions de certains use-cases principaux. Correction du planning
	& 0 %recherche
	& 0 %dev
	& 6 %reporting
	& 0\\ %admin
	
	14.05.2020
	& Rendez-vous avec M. Kapfer qui a mené à un changement de vision sur le projet. Utilisation de la technologie VNC afin d'éviter les vérifications côté client. Retour sur le planning qui semble trop chargé.
	& 0 %recherche
	& 0 %dev
	& 2 %reporting
	& 2\\ %admin
	
	27.05.2020
	& Recherche sur la communication VNC et un client Web. Définition d'un use-case permettant de définir la sécurité et la robustesse d'une potentielle architecture.
	& 4 %recherche
	& 0 %dev
	& 2 %reporting
	& 1\\ %admin
	
	28.05.2020
	& Présentation du use-case. Définition des technologies à utiliser lors de cette phase. Définition d'un plan à suivre pour la suite du projet. Discussion avec M. Rossier sur le rapport intermédiaire et sur la suite du projet.
	& 0 %recherche
	& 0 %dev
	& 4 %reporting
	& 2\\ %admin
	
	03.06.2020
	& Création du canevas du rapport intermédiaire.
	& 0 %recherche
	& 0 %dev
	& 5 %reporting
	& 1\\ %admin
	
	04.06.2020
	& Discussion avec M. Rossier sur la structure et les changements à y apporter.
	& 0 %recherche
	& 0 %dev
	& 1 %reporting
	& 1\\ %admin
	
	15.06.2020
	& Rédaction du rapport intermédiaire.
	& 0 %recherche
	& 0 %dev
	& 6 %reporting
	& 1\\ %admin
	
	16.06.2020
	& Rédaction du rapport intermédiaire.
	& 0 %recherche
	& 0 %dev
	& 6 %reporting
	& 0\\ %admin
	
	17.06.2020
	& Rédaction du rapport intermédiaire.
	& 0 %recherche
	& 0 %dev
	& 7 %reporting
	& 0\\ %admin
	
	18.06.2020
	& Rédaction du rapport intermédiaire.
	& 0 %recherche
	& 0 %dev
	& 9 %reporting
	& 1\\ %admin
	
	19.06.2020
	& Rédaction du rapport intermédiaire.
	& 0 %recherche
	& 0 %dev
	& 6 %reporting
	& 0\\ %admin
	
	15.06.2020
	& Rédaction du rapport intermédiaire.
	& 0 %recherche
	& 0 %dev
	& 6 %reporting
	& 1\\ %admin
	
	22.06.2020
	& Création du Dockerfile pour Mathematica. Beaucoup de complexité sur cet outil, car l'installation se fait depuis un ISO. Il faut encore trouver le moyen de mettre en place un moyen de monter l'ISO. Pour le moment, l'installation marche mais pas encore connecté au XScreen et pas de volume pour la persistance.
	& 0 %recherche
	& 8 %dev
	& 0 %reporting
	& 0\\ %admin
	
	23.06.2020
	& Fin de la création du Dockerfile. Il reste à rajouter une persistance sur le container mais tout marche bien. Le GUI se lance et il est possible de travailler sur Mathematica et sauvegarder des fichiers sur le container. La création du volume permettra de pouvoir écrire sur le filesystem natif. Le SI fournira une VM CentOs, mais pas de date de fixée pour le moment donc prise de retard de ce côté. Du côté client, un template a été défini et sera travaillé demain. Un article a été écrit sur Discourse et d'autres seront rédigés au fur et à mesure de l'avancement du travail. Le serveur n'a pas pu être configuré car la personne qui s'en charge  aux SI n'a pas pu encore le faire. De plus, la version fournie de Mathematica n'était pas sous licence valable et un temps considérable a été perdu avant de trouver la cause. Le SI avait mis à disposition une version ultérieure à celle prise en charge ... 
	& 0 %recherche
	& 8 %dev
	& 0 %reporting
	& 1\\ %admin
	
	24.06.2020
	& Création du canvas du site. Les fonctionnalités principales sont énoncées et visibles mais aucune connexion au niveau du serveur. Le SI n'a toujours pas fait suite à la demande des specs de la VM...
	& 0 %recherche
	& 4 %dev
	& 0 %reporting
	& 1\\ %admin
	
	25.06.2020
	& Retour de M. Rossier sur le rapport intermédiaire. Le détail de ce retour se trouve dans le dossier de PV  à la même date qu'aujourd'hui. Note de 4 sur le rapport intermédiaire. Démonstration du travail effectué. Présentation du container Docker avec Mathematica et du Website sans aucun backend. Demande de M. Rossier de revoir le planning
	& 0 %recherche
	& 0 %dev
	& 0 %reporting
	& 1\\ %admin
	
	26.06.2020
	& Refonte du planning et documentation de la containerisation.
	& 0 %recherche
	& 0 %dev
	& 4 %reporting
	& 2\\ %admin
	
	29.06.2020
	& Configuration de l'environnement Ubuntu pour la distribution. Suppression de beaucoup de bloatwares, il en reste encore. Écriture du script permettant de configurer les différents services de l'école. Après réunion avec GAPS, le bot ne sera pas utilisé pour le moment, une deuxième aura lieu dans quelques temps. Écriture d'un script de dépendances à enlever et à rajouter pour la suite.
	& 1 %recherche
	& 4 %dev
	& 1 %reporting
	& 1\\ %admin
	
	30.06.2020
	& Création du Dockerfile pour Logisim. Il existe une particularité sur ce container par l'utilisation de Java. Il faut rajouter \com{docker} à l'environnement graphique \com{xhost} manuellement. Après réunion avec M. Rossier, il a fallu refaire le planning et j'ai eu un feedback plus complet sur le rapport intermédiaire.
	& 1 %recherche
	& 5 %dev
	& 1 %reporting
	& 1\\ %admin
	
	01.07.2020
	& Travail à 100\% sur le script permettant de se connecter sur les différents services de l'école. Réussite de connexion au réseau HEIG-VD et eduroam. Création de la connexion VPN.
	& 1 %recherche
	& 4 %dev
	& 1 %reporting
	& 1\\ %admin
	
	02.07.2020
	& Finir le script permettant de centraliser. Discussion avec Alexandre Duc pour m'éclairer sur le stockage de credentials. Il faut plutôt utiliser le keyring de Ubuntu et pas refaire une sécu dessus. L'utilisation de LDAPs est suffisante dans le cadre pratique. En théorie, il faut éviter -> à documenter. La documentation pour eduroam est fausse au SI. Il faut aller sur le site de eduroam (officiel) et prendre le script d'installation de eduroam. Pour le moment, le réseau a été configuré sur le modèle de l'école mais il faut changer ça 
	& 2 %recherche
	& 5 %dev
	& 1 %reporting
	& 1\\ %admin
	
	03.07.2020
	& Rédaction du travail de la semaine
	& 2 %recherche
	& 0 %dev
	& 5 %reporting
	& 0\\ %admin
	
	06.07.2020
	& Finalisation du script de centralisation. Ajout de certaines conditions et modularisation du script- J'ai essayé de configurer un rendu graphique avec le serveur fourni par l'école, avec l'aide de David Truan, mais impossible. Le GUI ne se charge pas via \com{ssh}, ni via VNC. Du coup, j'ai pris la décision de travailler sur une VM. Afin de pouvoir avancer dans mon travail, on simulera un serveur avec une VM. Du côté distribution, j'ai essayé de générer un ISO mais pas encore réussi. Je continuerais ce travail plus tard.
	& 2 %recherche
	& 4 %dev
	& 1 %reporting
	& 0\\ %admin
	
	07.07.2020
	& Démonstration du script le matin à Daniel Rossier et explication de la situation. Essai de VNC sur VM, mais CentOs bloque trop de paramètres Docker et VNC. J'ai donc essayé de configurer quelque chose de performant mais aucun résultat.
	& 3.5 %recherche
	& 3.5 %dev
	& 1 %reporting
	& 0\\ %admin
	
	07.07.2020
	& J'ai donc essayé différentes distribs afin de trouver la plus adéquate à cette tâche. A priori, Manjaro, basé sur Arch, serait le choix que je vais faire. En effet, il y a toujours moyen de mettre en place un système d'ACL et Docker marche correctement. Ubuntu Server ne permet pas de mettre en place facilement les ACL ou le VNC. Manjaro permet de faire le tout sans trop de soucis à priori. 
	& 2 %recherche
	& 5 %dev
	& 1 %reporting
	& 0\\ %admin
	
	08.07.2020
	& J'ai réussi à mettre en place un serveur VNC et le lire avec le client VNC sur un ordinateur. Reste maintenant à comprendre comment mettre en place ça sur un site Internet. La technologie que j'avais trouvé ne permet pas de créer des interactions avec l'ordinateur diffusé. Il existe un "framework" qui permet de diffuser des flux sur un site Internet "Apache Guacamole". Cette technologie est à étudier mais permet de faire en sorte qu'un utilisateur peut interagir avec un ordinateur distant.
	& 6 %recherche
	& 2 %dev
	& 1 %reporting
	& 0\\ %admin
	
	09.07.2020
	& Apprentissage de Guacamole afin de mettre en place une connexion VNC avec des interactions sur l'ordinateur diffusé.
	& 5 %recherche
	& 4 %dev
	& 1 %reporting
	& 0\\ %admin
	
	10.07.2020
	& Ecriture du travail de la semaine 
	& 0 %recherche
	& 0 %dev
	& 6 %reporting
	& 0\\ %admin
	
	13.07.2020
	& Setup du serveur avec une base de données. La base de données renferme toutes les informations des utilisateurs de l'école. Il faut encore tester la robustesse du script avec quelques milliers de données mais le comportement global du script est le suivant : Récupération des données avec une requête SQL -> Parsing des données afin de créer un userspace pour chaque nouvel utilisateur, affectation à un groupe selon le rôle et création des dossiers pour chaque application. Chaque application est construite de base et depuis le site, un script sera appelé et lancera le container.
	& 2 %recherche
	& 6 %dev
	& 1 %reporting
	& 0\\ %admin
	
	14.07.2020
	& Test du script. Cleanup et optimisation au maximum pour le moment. Il reste encore quelques tests et c'est tout bon. Début de setup du serveur Guacamole. Pas mal de complications et ça ne marche toujours pas.
	& 3 %recherche
	& 5 %dev
	& 1 %reporting
	& 0\\ %admin
	
	15.07.2020
	& Setup continué de guacamole et j'ai réussi à le déployer. Le soucis reste la connexion VNC. En local, avec un VNCViewer, ça marche sans problème mais pas sur Guacamole
	& 0 %recherche
	& 0 %dev
	& 6 %reporting
	& 0\\ %admin
	
	16.07.2020
	& Contact de Loic Haas pour lui demander de l'aide sur Guacamole. Du coup en l'attendant, j'ai travaillé un peu sur le rapport. J'ai commencé les corrections par rapport au rendu intermédiaire afin de pouvoir partir sur une bonne base pour le rapport final. Après discussion, il a été possible de lancer et de faire tourner le framework.
	& 3 %recherche
	& 3 %dev
	& 3 %reporting
	& 0\\ %admin
	
	16.07.2020
	& Rédaction du rapport
	& 0 %recherche
	& 0 %dev
	& 6 %reporting
	& 0\\ %admin
	
\end{longtable}


\end{landscape}
