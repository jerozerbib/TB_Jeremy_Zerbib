\chapter{Décisions prises}

% \lipsum[1]


% \section{images ?}

% Voici comment mettre une image.

% \begin{figure}
% 	\centering
% 	\includegraphics[width=8cm]{TB/images/PGP_101.png}
% 	\caption{Schéma PGP}
% 	\label{fig:pgp}
% \end{figure}


% \section{comment citer une référence bibliographique ?}

% Ceci est un exemple de citation d'un livre de Pasini~\cite{pasini2015}.

% Mais aussi du site Web de Black Alps 2019~\cite{BA19} !


% \section{comment faire une référence ?}

% On peut aussi ajouter une référence à la section~\ref{sec:shell}.

% On peut aussi ajouter une référence à l'introduction, chapitre~\ref{ch:intro}.

% Comme montre la Figure~\ref{fig:pgp}, on peut référencer une figure.


% \section{comment afficher une commande simple ou du bash}

% Utiliser la commande \com{com}.

% Exemple : Test d'une commande bash shell \com{ls} : 

% Utiliser l'environnement \com{shellcmd}.

% \begin{shellcmd}
% $> ls -al test_underscore $$* "coucou"
% \end{shellcmd}
% Bli bLa


% \section{Commande Shell}
% \label{sec:shell}

% Voici comment faire une mise en forme de commande SHELL.

% Utiliser l'environnement \com{listingsbox}.
% \begin{description}
%  \item[1er paramètre:] le ype de shell, ici "console"
%  \item[2ème paramètre:] le nom à donner à la box
% \end{description}

% \begin{listingsbox}{console}{Exemple de commande shell avec réponse}
% root@kali:~$ bunzip2 data-decrypted.bin
% bunzip2: Can't guess original name for data-decrypted.bin -- using
% data-decrypted.bin.out
% \end{listingsbox}


% \section{Code inclus en direct dans le latex}

% Utiliser l'environnement \com{sourcebox}.
% \begin{description}
%  \item[1er paramètre:] le ype de shell, ici "c"
%  \item[2ème paramètre:] le nom à donner à la box
% \end{description}

% \begin{sourcebox}{c}{Exemple de code C}
% #include <stdio.h>

% int main(int argc, char* argv[])
% {
%   printf("Hello World!\n");
%   return 0;
% }
% \end{sourcebox}



% \section{Code à partir d'un fichier}

% Utiliser la commande \com{inputsourcecode}.
% \begin{description}
%  \item[1er paramètre:] le ype de shell, ici "c"
%  \item[2ème paramètre:] le nom du fichier source, "\path{source_code/example.c}"\\ Egalement exemple de la commande \com{path}.
%  \item[3ème paramètre:] le nom à donner à la box
% \end{description}


% %% Inclure du code source d'un fichier externe
% %% 1er paramètre: langage du code
% %% 2ème paramètre: le path du fichier source
% %% 3ème paramètre: le titre de la box
% \inputsourcecode{c}{"source_code/example.c"}{example.c}

\section{Distribution}
\section{Gestion de licenses}

