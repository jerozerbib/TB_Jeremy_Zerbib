\chapter{Cahier des charges}



\section*{Résumé du problème}
Le but de ce travail est de créer une distribution Linux pour les étudiants en \textit{TIC} et \textit{TIN}, de manière à sensibiliser les étudiants de l'école à  l'importance de maîtriser un outil tel que Linux dans le cadre de l'ingénierie.
Beaucoup d'aspects quant à l'utilisation de Linux sont impératifs pour les cours des deux orientations et aucune formation officielle n'est donnée dans le cadre des cours. 
L'utilisation d'outils comme \textit{docker} ou \textit{QEMU} sont nécessaires à certaines matières enseignées mais peu d'étudiants savent les utiliser. 
En initiant les étudiants dès la première année, en fournissant un système complet et malléable, il est possible de combler ce manque, déjà remarqué par certains professeurs.
Le changement de plan d'études propose une transition vers ce système sans modifier le fonctionnement actuel d'enseignement  au profit des étudiants.
\newline 
Une distribution configurée de manière optimale permettrait aux étudiants de suivre les cours de la HEIG-VD de manière pérenne et sécurisée si une couche de sécurité est rajoutée avec des sauvegardes sécurisées et automatique, des pare-feux avec des configurations proposées, des VPN installés, etc. 
\newline
Il serait donc intéressant de mettre à disposition une plateforme qui permette de garantir l'utilisation de manière licite des logiciels payants, fournis par l'HEIG.
En effet, les logiciels fournis par l'école sont soumis à des conditions générales d'utilisation qui demandent à ce que le logiciel soit utilisé seulement dans le cadre de l'école.

\subsection*{Problématique}
Ce travail permettrait de créer une infrastructure permettant l'installation, la configuration et le déploiement des mises à jour ainsi que de générer des sauvegardes automatiques. 
Tout cela de manière très stricte et sécurisée.
Cet aspect de ce travail est rempli avec la distribution Linux.
\newline 
De plus, une gestion des licences doit être mise en place de façon à assurer l'utilisation de divers programmes de manière non commerciale. 
\newline 
Finalement, la création d'une telle infrastructure permettrait d'optimiser certains aspects de la vie des étudiants.
\newline Pour résumer, nous pouvons résumer ce projet en deux objectifs distincts : 

\begin{itemize}
    \item Création d'un installateur permettant de mettre en place un environnement de travail adéquat pour les étudiants de l' \textbf{HEIG-VD} 
    \item Création d'une gestion de l'ouverture d'un logiciel payant fourni par l'école ainsi que les interactions avec un filesystem donné
\end{itemize}

\subsection*{Solutions existantes}
La solution proposée actuellement aux deux objectifs sont respectivement :
\begin{itemize}
    \item Mise à disposition de différentes machines virtuelles suivant les matières. 
    Cette solution est efficace mais demande beaucoup d'espace disque pour un étudiant.
    De plus, suivant les matières, il y a beaucoup de répétitions entre les machines virtuelles installées
    \item À ce jour, \textit{eistore} est la seule manière d'obtenir légalement les logiciels fournis par l'école. 
    Une fois installés, il n'y a aucune forme de contrôle sur l'utilisation de ces derniers.
\end{itemize}

\subsection*{Solutions exploratoires}
\begin{itemize}
    \item D'un point de vue distribution, un dérivé de la dernière version d'Ubuntu paraît la meilleure option.
    \item Pour la gestion de licences, l'utilisation d'un portail Web utilisant la technologie \textit{\acrfull{vnc}} est la solution qui sera explorée lors de ce travail.
\end{itemize}

\section*{Cahier des charges}
La liste ci-dessous explicite les tâches qui ont été effectuées lors de ce travail, avec un pourcentage indicatif du temps passé sur chacune d'elles :
\begin{enumerate}
    \item Recherche sur les différentes distributions Linux afin de trouver la meilleure base possible (5\%)
    \item Recherche sur les différentes techniques de containerisation et de la gestion des interactions entre le filesystem et l'application (10\%)
    \item Conception d'une interface Web permettant de faire des tests  avec le tunnel \acrshort{vnc} (20\%)
    \item Création et configuration d'un serveur distant permettant de lancer des applications via un container Docker (10\%) 
    \item Liaison entre la plateforme de tests et le serveur distant (20\%)
    \item Preuve de la robustesse de la plateforme à travers des séries de tests qui seront automatisés (5\%)
    \item Configuration et mise en place de la distribution (15\%)
    \item Preuve et tests sur la distribution afin de la mettre en production (5\%)
    \item Rédaction du rapport, documentation et mise à dispositions des plateformes (système d'exploitation et des logiciels), présentation du travail (10\%)
\end{enumerate}


\subsection*{Objectifs}
Les objectifs de ce travail sont les suivants :
\begin{itemize}
    \item Fournir un système qui gère et garantit l'utilisation des licences de manières licites
    \begin{itemize}
        \item Garantie de l'authentification d'un utilisateur
        \item Garantie de la portabilité de la plateforme sur tous systèmes d'exploitation
        \item Garantie de l'isolation de l'application qui implique une garantie de l'utilisation licite de cette dernière
    \end{itemize}
    \item Fournir un système d'exploitation suffisamment optimisé pour simplifier les études d'un étudiant
    \begin{itemize}
        \item Système d'exploitation basé sur Ubuntu 20.04
        \item Automatisation de la connexion à tous les services de l'école
        \item Documentation permettant d'aider à la configuration et à la manipulation d'un système Linux dans le cadre de l'école.
    \end{itemize}
\end{itemize}

% \subsection*{Déroulement}

\subsection*{Livrables}
Les livrables seront les suivants :
\begin{enumerate}
\item Une documentation contenant :
	\begin{itemize}
	\item Les détails des recherches effectuées
	\item Les décisions découlant de ces recherches
	\item Les détails concernant le changement de direction pris au milieu du travail
	\item L'architecture utilisée pour la gestion de licenses et la distribution 
	\item Les informations concernant le fonctionnement et les limitations de la plateforme Web 
	\item Les informations concernant le fonctionnement et les limitations de la distribution 
	\item Une planification initiale et finale
	\item Un mode d’emploi d'utilisation de distribution
	\end{itemize}
\item Une plateforme Web regroupant les logiciels utilisable par un élève
\item Un système d'exploitation sous la forme d'un \textit{.iso} à télécharger sur les serveurs de l'école
\end{enumerate}

